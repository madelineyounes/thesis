\chapter{Introduction}\label{ch:intro}
\section{Problem Statement}\label{sec:problem}
Although voice enabled technology is used globally advancements have not benefited 
all languages equally. With languages other than English that have a diverse range of dialects limited in 
advancements due to data scarcity and the complexity that comes from shared 
features between the dialects. One language that this problem has  become apparent is
Arabic. In multicultural societies such as that in Sydney, NSW it is common to have speakers of different dialects interact. 
There are around 200 thousand Arabic speakers within NSW, majority speaking Levantine dialect of Arabic with then a spread among Egyptian and Gulf Arabic.
It would be a common scenario where perhaps for a telehealth consultation
the only Arabic speaker available doesn't speak the same dialect as a patient.
The speaker may then code switch with the words, phrases etc. they know from one dialect to their main dialect to hold a conversation. 
Having a system that is able to accurately identify the dialects spoken in certain segments to then produce an accurate transcription would certainly be helpful in that scenario. 
Dialectal Arabic generally has very limited available datasets, with no commercially available datasets and therefore is considered low resource. Current methods use phoneme recognition or 
traditional machine learning but both these methods have flaws that limit their ability to reliably recognise Arabic dialects. 
As phoneme identifiers rely on phonemic differences between the dialects, where there are shared phonemes distinguishing the dialects become a difficult task.
While traditional machine learning requires large amounts of labelled training data which is currently unavailable for Arabic dialects.  
In most voice enabled technology the first step in selecting an automated speech recognition model (ASR) is to first 
identify its language or dialect. Dialectal Identifiers enable more accurate systems in speech recognition and transcription by allowing 
a speech models tuned for a particular dialect to be selected at the start of this process. In order to build a high performance 
Arabic dialectal identifier (DID) desirable methods should require low amounts of data and leverage existing speech models. 
However, no work has been done using this modern method of transfer learning and so there is a need to apply contemporary techniques to Arabic dialectical identifiers (DID). 
This thesis will investigate transfer learning as an approach in designing more accurate and reliable Arabic DIDs. 
The dialectal identifier should accurately distinguish the four umbrella Arabic 
dialects, and then it's robustness tested by extending it to a finer grain downstream task of distinguishing seventeen regional dialects. 


\section{Thesis Aims}
The goal of this thesis is to answer the overarching question:\emph{
Can transfer learning be leveraged to be used to improve the performance of the low resource task of Arabic Dialectal Identification?}\\
\\The key aims of this thesis are: 

\begin{itemize}
    \item To assess the viability of using transfer learning to improve the accuracy of Arabic \\Dialectal Identification. 
    \item To investigate the performance of a transfer learning based DID on low resource dialects. Through assessing the minimum amount of data needed to accurately fine tune a DID system. This is in terms of utterance length of the audio files and the amount of files provided. 
    \item To critically analyse which pretrained model and downstream model architecture is able to produce the most accurate DID system. 
    \item To assess the effect of unfreezing and fine-tuning the encoder layers within the pretrained model on the performance of the DID system. 
    \item To explore whether a transfer learning based DID system can be adapted to be accurately applied to a finer set of dialectal groups. 
\end{itemize}

\section{Chapter Outline}
This report is organised as described. 
\begin{itemize}
\item Chapter~\ref{ch:background} details some background information surrounding the impacts of this thesis and describes the unique features of Arabic Dialects. 
\item Chapter~\ref{ch:litreview} provides a detailed analysis of current LID and DID methodologies. As well as 
details the literature which supports the use of transfer learning for the application of LID and DID. 
\item Chapter~\ref{ch:methodologies} proposes a methodology for this thesis. 
\item Chapter~\ref{ch:experiment} explores the experimentation conducted throughout this thesis and analyses the results. 
\item Chapter~\ref{ch:conclusion} draws up conclusions and summarises the key takeaways of the report. As well as providing suggestions for future work and the implications of this thesis on similar research. 
\end{itemize}
