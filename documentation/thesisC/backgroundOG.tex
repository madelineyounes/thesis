\chapter{Background}\label{ch:background}
To define Language Identification (LID), it is the process of differentiating spoken audio and 
classifying the audio segment based on its corresponding language. 
LID systems are the critical first step in selecting the most accurate model to 
use for Automatic Speech Recognition (ASR), multilingual transcription or other automated speech processing systems. 
A dialect is a sub-variant of a language which is usually mutually intelligible by other speakers of that 
language despite the speaker using a different dialect. Dialects evolve within a certain region, area 
or within a class. Dialect Identification (DID) generally posses some interesting challenges compared to language 
identification that semi-self supervised systems could provide solutions to. These challenges include that 
dialects unlike languages are not standardised, generally have very limited labelled datasets and 
so, are often considered low resource and the differences between different dialects are often not as clear 
as the differences between languages. 


\section{Arabic Dialects}
This thesis will focus on Arabic dialects as despite being a widely spoken language and dialects 
being the primary spoken form of Arabic, there are still significant 
improvements that can be made to Arabic DIDs with current systems achieving the highest accuracy of 86.29\%. \\

Arabic is the official language of 25 countries and has 330 million native speakers.
\\Academically the regional dialects are usually grouped into 5 main groups North African (NOR), Egyptian (EGY), Levantine (LEV), Gulf (GLF) and Modern Standard Arabic (MSA). 
MSA is taught academically in most Arabic speaking countries and originates from the Gulf region but is not used 
for general conversation or outside academic setting.


Comparatively to MSA, the lack of standardisation in dialectal Arabic has resulted in more linguist complexities.
Dialectal Arabic has a richer morphology and cliticisation system, accounting for circumfix negation, 
and for attached pronouns to act as indirect objects. As well as this some words are shared but are used for
differing functions eg. For example,'Tyb' is used as an adjective in MSA but dialectal as an interjection. 
North African has the largest amount of dialectical variation within the dialect and is the most different from the other Arabic dialects. Taking influences from French and Berber languages. 
Egyptian globally is the widely understood dialect due to the Egyptian movie and television industry. 
Levantine dialects differ slightly in pronunciation and intonation but are equivalent when transcribed. Closely related to Aramaic. 
Gulf is the form which is most closely related to MSA and preserves many of MSA verb conjugations. 
Understanding of different dialects depends on an individuals exposure outside their own country. eg. due to the prevalence of Egyptian television and movies,
many Arab people can understand the Egyptian Dialect but a Levantine speaker would not be able to understand 
the Moroccan dialect. 

The differences between Arabic dialects are comparable to the differences present in North Germanic languages such as
Norwegian, Swedish, Danish or the West Salvic languages eg. Czech, Slovak, Polish. Some linguistic variation between dialectal Arabic include incongruous morphemes, prepositions
verb conjugations, word meanings, phonemes and pronunciation. Some examples of this is shown in the table \ref{tab:dialectDifferences}. [15,16]\\

In addition to this majority of available pretrained models that are used in self supervised or semi supervised systems are trained on 
English datasets. Arabic has 6 vowels/diphthongs in MSA and 8-10 vowels in most dialects, 
28 constants while English has 24 consonants and 22 vowels. [59]
As well as this compared to English, Arabic and particularly dialectal Arabic has large amount of morphemes, rendering 
it unfeasible for the training data to contain all the possible morphemes. So, a system which has been built to operate
well for a DID that is linguistically different to English should thereby be robust enough to be applied to other languages with similar complexity.\\ 
Hence, the two key challenges in creating an Arabic DID which will be explored in this thesis are: 
\begin{itemize}
    \item Dialectal Arabic is considered low resource, as there are no large commercially available datasets. 
    \item There is a significant amount of complex linguist differences and similarities between Arabic dialects. 
\end{itemize}
More details about the ADI17 dataset to be used are provided in Chapter \ref{ch:prelim}, 
the dataset contains audio from 17 countries and be divided into the 4 widely spoken major dialect groups (NOR, EGY, LEV, GLF), 
as MSA is not spoken in general conversation it is not included in the dataset and will not be identified by the DID designed in this thesis.\\


\begin{table}[htb]
    \begin{center}
    \begin{tabular}{|c || c | c | c | c|}
    \hline
    English/Feature & MSA & LEV & GLF & EGY \\ [0.5ex] 
    \hline\hline
    Money & nqwd & mSAry & flws & flws \\ 
    \hline
    I want & Aryd & bdy & Ab\textgammaý & ςAyz \\
    \hline
    Now & AlĀn & hlq & AlHyn & dlwqt \\
    \hline
    When? & mtý? & Aymtý? & mtý? & Amtý \\
    \hline
    alveolar affricate sound & dj & j & y & g \\
    \hline
    Handsome & djami:l & jami:l & yami:l & gami:l \\
    \hline
    consonant sound & \textTheta & t & \textTheta & t \\
    \hline
    Three & {\textTheta}ala:{\textTheta}a & tla:te  & {\textTheta}ala:{\textTheta}a  & tala:ta \\[0.5ex] 
    \hline
    \end{tabular}
    \caption{Examples of linguistic differences between Arabic Dialects}
    \label{tab:dialectDifferences}
    \end{center}
\end{table}