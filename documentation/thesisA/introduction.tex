\chapter{Introduction}\label{ch:intro}
\section{Problem Statement}\label{sec:problem}
In multicultural societies such as that in Sydney, NSW it is common to have speakers of different dialects interact. 
There are around 200 thousand Arabic speakers within NSW, majority speaking Levantine dialect of Arabic with then a spread among Egyptian and Gulf Arabic.
It would be a common scenario where perhaps for a telehealth consultation
the only Arabic speaker available doesn't speak the same dialect as a patient.
The speaker may then code switch with the words, phrases etc. they know from one dialect to their main dialect to hold a conversation. 
Having a system that is able to accurately identify the dialects spoken in certain segments to then produce an accurate transcription would certainly be helpful in that scenario. 
Dialectal Arabic has limited available datasets, no commercially available datasets and is considered low reasource. Current methods use phoneme recognition or 
traditional machine learning but both these methods have flaws that limit their ability to reliably recognise Arabic dialects. 
As phoneme identifiers rely on phonemic differences between the dialects, where there are shared phonemes distinguishing the dialects become a difficult task.
While traditional machine learning requires large amounts of labelled training data which is currently unavailable for Arabic dialects.  
This thesis will investigate an accurate and reliable method to segment conversations of dialectal Arabic into
homogenous dialectal segments and identify the dialect of each segment.
The dialectal identifier should accurately distinguish the four major Arabic 
dialects and then further extended to be able to distinguish between seventeen regional dialects. 




\section{Thesis Aims}
The goal of this thesis is to answer the overarching question:\emph{
How can the existing limited resource Arabic dialect speech Corpa be used to improve the accuracy
Arabic Dialectal \\Identification? }\\
\\The key aims of this thesis are: 

\begin{itemize}
    \item To assess the viability of using transfer learning to improve the accuracy of Arabic \\Dialectal Identification. 
    \item To investigate the performance of a transfer learning based DID on low resource dialects. Through assessing the minimum amount of data needed to accurately fine tune a DID system.
    \item To critically analyse which pretrained model and downstream model architecture is able to produce the most accurate DID system. 
    \item To explore whether a transfer learning based DID system can be adapted to be accurately applied to a finer set of dialectal groups. 
\end{itemize}

\section{Chapter Outline}
This report is organised as described. 
Chapter~\ref{ch:background} details some background information surrounding the impacts of this thesis and describes the unique features of Arabic Dialects. 
Chapter~\ref{ch:lit review} provides a detailed analysis of current LID and DID methodologies. As well as 
details the literature which supports the use of transfer learning for the application of LID and DID. 
Chapter~\ref{ch:outline} proposes a methodology for this thesis and details the expected results.
Chapter~\ref{ch:prelim} explores the preliminary work conducted for this thesis. 
Chapter~\ref{ch:plan} details the work to be conducted during this thesis and outlines the possible challenges
Chapter~\ref{ch:conclusion} draws up conclusions and summaries the key takeaways of the report. 

